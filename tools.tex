% -*- root: main.tex -*-
% 1) pdflatex main
% 2) makeindex main.idx -s StyleInd.ist
% 3) biber main
% 4) pdflatex main x 2
%-------------------------------------------------------------------------------
\chapterimage{chapter_head_1.pdf}

%-------------------------------------------------------------------------------
\chapter{ROS Tools}

%-------------------------------------------------------------------------------
\section{ROS Tools (RVIZ)}\index{ROS Tools (RVIZ)}

%-------------------------------------------------------------------------------
\subsection{Tools}\index{Tools}

%-------------------------------------------------------------------------------

\begin{enumerate}
\item The first item
\item The second item
\item The third item
\end{enumerate}

\begin{enumerate}
  \setcounter{enumi}{4}
  \item fifth element
\end{enumerate}

\begin{enumerate}
  \item The first item
  \begin{enumerate}
    \item Nested item 1
    \item Nested item 2
  \end{enumerate}
  \item The second item
  \item The third etc \ldots
\end{enumerate}


\begin{itemize}[leftmargin=*]
\item The first item
\item The second item
\item The third item
\end{itemize}

\begin{corollary}[Corollary name]
asdfasdfasdfasdfasdf
\end{corollary}

\begin{description}
\item[Name] Description
\item[Word] Definition
\item[Comment] Elaboration
\end{description}


\begin{description}
  \item[First] \hfill \\
  The first item
  \item[Second] \hfill \\
  The second item
  \item[Third] \hfill \\
  The third etc \ldots
\end{description}


\begin{table}[htp]
\centering
\begin{tabular}{l l l}
\toprule
\textbf{Treatments} & \textbf{Response1} & \textbf{Response2}\\
\midrule
Treatment 1 & 0.0003262 & 0.562 \\
Treatment 2 & 0.0015681 & 0.910 \\
Treatment 3 & 0.0009271 & 0.296 \\
\bottomrule
\end{tabular}
\caption{Tablecaption}
\end{table}

%------------------------------------------------

\section{Figure}\index{Figure}

\begin{figure}[htp]
\centering
\includegraphics[scale=0.5]{picture}
\caption{caption}
\end{figure}

\begin{figure}[htp]
\centering
\includegraphics[width=0.5\columnwidth]{picture}
\caption{caption}
\end{figure}



\vspace{\baselineskip}

\setcounter{num}{0}

\vspace{\baselineskip}
\noindent
\stepcounter{num}
\thenum


\setcounter{num}{0}
\vspace{\baselineskip}
\noindent
\stepcounter{num}\circled{\thenum} テクストをいれる。
\stepcounter{num}\circled{\thenum} 確認する。


→←↑↓

\textcolor{green}{緑}
{\color{limegreen}緑}
\textcolor{orange}{オレンジ}
\textcolor{red}{赤}


\vspace{\baselineskip}
\begin{lstlisting}[language=ROS]
%*
*)
\end{lstlisting}

\begin{lstlisting}[language=C++]
\end{lstlisting}
\begin{lstlisting}[language=bash]
\end{lstlisting}
\begin{lstlisting}[language=XML]
\end{lstlisting}
\begin{lstlisting}[language=make]
\end{lstlisting}
\begin{lstlisting}[language=YAML]
\end{lstlisting}

%*コメント*)

\textbf{}

double quotes
`` ''
single quotes
` '

\begin{exercise}[ROSシェルコマンドを使用できる環境]
\end{exercise}

\setcounter{num}{0}

\stepcounter{num}\circled{\thenum}

\cite{book_key}

\textbf{セクション~\ref{sec:RosTerm}~\nameref{sec:RosTerm}(pp.\pageref{sec:RosTerm})}

\textasciitilde

\begin{center}
ROSは \textbf{メタ・オペレーティングシステム(Meta-Operating System)}である。
\end{center}

\begin{itemize}
\item  rosclean [オプション]
\end{itemize}

\vspace{\baselineskip}
\noindent
\begin{description}
\item[]
\end{description}

%-------------------------------------------------------------------------------
\chapter{ROSツール}\index{ROSツール}

%-------------------------------------------------------------------------------
\section{3次元可視化ツール}\index{3次元可視化ツール}

%-------------------------------------------------------------------------------
\subsection{RVizのインストールと実行}

\subsubsection{SBCやMCUとの接続}

%-------------------------------------------------------------------------------

\footnote{\url{URL}}

\tiny
\scriptsize
\footnotesize
\small
\normalsize
\large
\Large
\LARGE
\huge
\Huge

\begin{lstlisting}[moredelim={[is][keywordstyle]{@@}{@@}}]
source /opt/ros/@@indigo@@/setup.bash
\end{lstlisting}

\noindent\fbox{%
  \parbox{\textwidth}{%
    ROS (Robot Operating System)はソフトウェア開発者のロボット・アプリケーション作成を支援するライブラリとツールを提供しています。具体的には、 ハードウェア抽象化、デバイスドライバ、ライブラリ、視覚化ツール、 メッセージ通信、パッケージ管理などが提供されています。
  }%
}\\

\begin{term}{title}
\end{term}

\begin{textbox}
\end{textbox}

\textbf{}
\\\\
\vspace{\baselineskip}
\noindent\textbf{}
\verb|~|
\label{section:xxxxx}
\textbf{\ref{section:xxxxx}}
\textbf{~\ref{section:xxxxx}~\nameref{section:xxxxx}(pp.\pageref{section:xxxxx})}
\textbf{\nameref{section:terms}(pp.\pageref{section:terms})}

1章
1.1節
1.1.1項
